\setcounter{page}{1}
\chead{\small\itshape Contributions and Statements}

\section{Contributions to Research and Development}

\subsection{Articles Published or Accepted in Peer-Reviewed Journals}

\begin{itemize}[~,noitemsep]
    \item \textbf{Findlay, W.}, Somayaji, A. B., and Barrera, D. (2020) bpfbox:
    Simple Precise Process Confinement with eBPF. Proceedings of the 2020 ACM
    Cloud Computing Security Workshop (CCSW'2020). (Master's work)
\end{itemize}

\setcounter{subsection}{3}
\subsection{Non-Peer-Reviewed Contributions}

\begin{itemize}[~,noitemsep]
    \item \textbf{Findlay, W.} (2021) A Practical, Flexible, and Lightweight Confinement
    Framework in eBPF. MCS thesis. (Master's work)
    \item \textbf{Findlay, W.}, Barrera, D., and Somayaji, A. B. (2021) BPFContain: Fixing the Soft Underbelly of Container Security. Archival pre-print, pending submission to Usenix Security 2022. (Master's work)
    % \item \textbf{Findlay, W.} (2020) bpfbox: Simple Precise Process Confinement
    % with eBPF. Conference presentation, presented at the 2020 ACM Cloud Computing
    % Security Workshop (CCSW'2020). (Master's work)
    \item \textbf{Findlay, W.} (2020) Host-Based Anomaly Detection with Extended
    BPF. Honours thesis. Carleton University. (Undergraduate work)
    % \item \textbf{Findlay, W.} (2020) Host-Based Anomaly Detection with Extended
    % BPF. Technical talk presented in seminar series at the Carleton Computer and
    % Internet Security Lab, Carleton University. (Undergraduate work)
\end{itemize}

\section{Most Significant Contributions to Research and Development}

\paragraph*{\textsc{bpfbox}: Simple Precise Process Confinement with eBPF}
This peer-reviewed research paper presents \textsc{BPFBox}\footnote{\textsc{BPFBox}
is free and open-source software, available under the GPLv2 license at
\url{https://github.com/willfindlay/bpfbox}}, the first process confinement
mechanism written in Extended Berkeley Packet Filter (eBPF).
Process confinement is critical for restricting access to security-sensitive resources on
our computers and reducing the attack surface for potential security exploits.
\textsc{BPFBox}'s advantages over existing process confinement solutions include a simple
yet expressive policy language and the ability to express and enforce policy across
userspace and kernelspace boundaries, something which no existing process confinement
mechanism can do. Furthermore, experimental data presented in the paper shows that
\textsc{BPFBox}'s performance is competitive with (and in some cases better than) the most
popular process confinement mechanisms in Linux.

In this work, I designed and implemented the \textsc{BPFBox} research prototype, including
the policy language and enforcement engine. As the first author of the research paper,
I conducted and presented the results of all the benchmarks and experiments presented in
the paper, described all of the project's technical details, and wrote significant
portions of the background material.  My co-authors, Dr.\ Anil Somayaji and Dr.\ David
Barrera, helped with the positioning of the work, writing up portions of the background
material, and selecting the appropriate venue for publication.

Our publication venue, the 2020 ACM Cloud Computing Security Workshop (CCSW)---a part of
ACM SIGSAC---is a top security workshop in cloud computing and presents an ideal
target audience for our experimental work in novel process confinement mechanisms.
Submissions to CCSW are competitive, with only a 30\% acceptance rate (12 papers out
of 40 submissions). This paper was presented in November 2020 at the workshop. I was also
invited to give a follow-up presentation at IBM research in December 2020.

\paragraph*{\textsc{BPFContain}: Fixing the Soft Underbelly of Container Security}
This paper presents the design, implementation, and evaluation
\textsc{BPFContain}\footnote{\textsc{BPFContain} is free and open-source software,
available under the GPLv2 license at \url{https://github.com/willfindlay/bpfbox-rs}.}, an
extension on top of the original \textsc{BPFBox} design with container security in mind.
In particular, we develop a method for incorporating container semantics into kernelspace
policy enforcement, using eBPF programs to trace the lifecycle of Linux containers. This
approach yields a surprisingly effective policy enforcement mechanism that simultaneously
simplifies container confinement policy and improves overall security.

Unlike individual processes, Linux containers group a set of processes and related
resources together, defining a semantic relationship between them and establishing a clear
boundary between resources within and without the container. By tracing the
container lifecycle and incorporating these relationships into the policy engine,
\textsc{BPFContain} cam enforce container-level policy at a significantly finer
granularity than would be possible with existing techniques. Moreover, the implicit
security boundary around the container obviates the need to specify access to resources
which only exist within the context of the container. This property radically simplifies
\textsc{BPFContain} policies compared to traditional approaches.

In this work, I designed and implemented the \textsc{BPFContain} research prototype as an
extension on top of my original \textsc{BPFBox} design. As the first author, I conducted all
of the experiments, wrote up all the technical details of the implementation. My
co-author, Dr.\ Anil Somayaji and Dr.\ David Barrera helped with positioning, provided
feedback on early drafts, and contributed editorial changes in various sections.
This work has been made available as a pre-print on the arXiv paper database, and we
plan to submit it for publication in Usenix Security 2022.


% \paragraph*{Host-Based Anomaly Detection with Extended BPF}

% In my undergraduate honours thesis, I presented the design and implementation of
% ebpH\footnote{ebpH is free and open source software, available under the GPLv2
% license at \url{https://github.com/willfindlay/ebph}}, a host-based anomaly detection system written in eBPF. As a successor to
% the original pH anomaly detection system written by Dr.~Anil Somayaji in 2002,
% ebpH is significant to the fields of operating system security and intrusion
% detection as the first modern implementation of system call sequence analysis
% for anomaly detection.

% As the sole author of this work, I designed and implemented the research
% prototype, conducted experiments to test ebpH's performance and functionality,
% and wrote up the technical details and challenges associated with ebpH's
% implementation.  My undergraduate thesis advisor, Dr.~Anil Somayaji, provided
% guidance where necessary and his original work on the pH anomaly detection
% system served as the basis for parts of my design.

% While undergraduate theses are not published, an abstract of my work was
% published in the Carleton Honours Project and Thesis repository. We intend to
% submit a paper covering a later version of ebpH to a top security conference
% later this year.  I presented ebpH at a technical talk given to members of the
% CISL and CCSL research groups at Carleton University in April 2020.

\section{Applicant's Statement}

\paragraph*{Research Experience}
My research in operating system security has afforded me a strong technical
knowledge of the underlying abstractions, security mechanisms, data structures,
and algorithms that power our computer systems. This technical understanding has
led me to question whether the security mechanisms that are currently in place
in most commodity operating systems are sufficient to protect our devices
against more sophisticated attacks. My experiences with using operating system
observability technologies to build both anomaly detection systems and process
confinement mechanisms has motivated me to consider whether it may be possible
to bridge the gap between adaptive security approaches and traditionally static
approaches like process confinement, a notion that has fundamentally informed
my future research directions.

\paragraph*{Relevant Activities}
I have been a teaching assistant for the COMP3000 (Operating Systems) and COMP4000
(Distributed Operating Systems) course at Carleton University for three years. During this
time, I have provided guidance for upper year computer science students and graduate
computer science students and taken an active role in the development of course material
and other administrative tasks. In particular, I was involved in the design of tutorials
and assignments for both COMP3000 and COMP4000. These focused on key aspects of operating
system design, implementation, and security.  The strong
interplay between my research and teaching activities has been quintessential in fostering
the growth of my academic career. As a direct result of my passion for operating system
security, I have been nominated for three consecutive Outstanding Teaching Assistant
awards in the 2018--2019, 2019--2020, and 2020--2021 academic years.

% In COMP4000, I have been involved in designing
% assignments focused around the configuration and deployment of a distributed Kubernetes
% cluster.  This experience has provided a strong basis for my research on cloud security.
% There is strong interplay between
% my teaching activities and the directions I have taken in my area of research, distributed
% operating system security; my knowledge of operating systems has been a driving factor in
% my research and I have attempted to bring aspects of my security background into my
% teaching.

In the final year of my undergraduate studies, I was selected to participate in the
Carleton School of Computer Science's Accelerated Master's Program. This option is only
offered to top undergraduate students in Carleton's Computer Science program. It allows
the student to take two graduate-level courses in the final year of their degree. This
opportunity has provided me with a strong background in research early on in my graduate
school career and enabled me to complete my Master of Computer Science degree in one year
instead of two.  After successfully defending my Master's thesis, I was nominated by my
committee for a senate medal for academic achievement.

As a member of the Carleton Internet Security Lab (CISL) at Carleton University
and a close collaborator with its sister lab, the Carleton Computer Security Lab
(CCSL), I have access to the resources, knowledge, and guidance of one of the
largest computer security research groups in Canada. This has proved to be
a significant advantage in the development and dissemination of my research thus
far and will continue to be an invaluable resource in my future research
endeavours.
